\documentclass{beamer}

\mode<presentation>
{
  \usetheme{Warsaw}
  \setbeamercovered{transparent}
}

\usepackage{amsmath}
\usepackage[english]{babel}
\usepackage[latin1]{inputenc}
\usepackage{times}
\usepackage[T1]{fontenc}
\usepackage{sagetex}
\usepackage{upgreek}

\title{An Introduction to Sage}

\author[Arvind]
{Arvind S Raj}

\institute[Amrita]
{
  Department of Cybersecurity Systems and Networks\\
  Amrita University, India
}

\date[FOSDEM 2014]
{1 February 2014 / FOSDEM}

\AtBeginSubsection[]
{
  \begin{frame}<beamer>{Outline}
    \tableofcontents[currentsection,currentsubsection]
  \end{frame}
}

\begin{document}

\begin{frame}
  \titlepage
\end{frame}

\begin{frame}{Outline}
  \tableofcontents
\end{frame}

\begin{frame}{Bio}
  \begin{itemize}
    \item Graduate CS student at Amrita University, India.
    \item Interested in computer security and Python.
    \item Used Sage to graduate labs in Cryptography and in CTF contests.
  \end{itemize}
\end{frame}

\begin{frame}{Introduction to Sage}
  \begin{itemize}
   \item GPL licensed mathematics software.
   \item Unified interface to about 90 popular Python libraries.
   \item Two modes: command(like Python shell) and notebook(web interface).
   \item IPython shell and Python programming language.
   \item ``sagerc`` file: \$HOME/.sage/init.sage or \$SAGE\_STARTUP\_FILE.
  \end{itemize}
\end{frame}

\begin{frame}[fragile]{Arithmetic and built-in functions}
  \begin{itemize}
   \item General arithmetic supported by an (I)Python shell.
   \begin{itemize}
    \item \textasciicircum \hspace{0pt} is exponent and \textasciicircum\textasciicircum \hspace{0pt} is XOR.
    \item For integers, / reduces to lowest fraction and // performs integer division.
   \end{itemize}
   \item Support mathematical functions and constants with arbitrary precision.
   \begin{itemize}
    \item \verb+pi.n(digits=20)+ = \sage{pi.n(digits=20)}
    \item \verb+e.n(digits=25)+ = \sage{e.n(digits=25)}
    \item \verb+golden_ratio.n(prec=60)+ = \sage{golden_ratio.n(prec=60)}
    \item \verb+n(sin(pi/3), prec=60)+ = \sage{n(sin(pi/3), prec=60)}
    \item \verb+sqrt(263).n(digits=20)+ = \sage{sqrt(263).n(digits=20)}
    \item \verb+n(cos(5*pi/4), prec=60)+ = \sage{n(cos(5*pi/4), prec=60)}
   \end{itemize}
  \end{itemize}
\end{frame}

\begin{sagesilent}
 var('x y t')
 c = circle((0, 0), 4)
 ring = IntegerModRing(12)
 def f(x,y):
  return math.sin(y*y+x*x)/math.sqrt(x*x+y*y+.0001)
\end{sagesilent}

\begin{frame}{Algebra}
  \begin{itemize}
    \item Factorizing polynomials.
    \begin{itemize}
      \item $factor(x^{4} - 15 x^{3} + 84 x^{2} - 208 x + 192)$ = $\sage{factor(x^4 - 15*x^3 + 84*x^2 - 208*x + 192)}$
      \item $factor(x^{3} - 6 x^{2} +11 x -6)$ = $\sage{factor(x^3 - 6*x^2 +11*x -6)}$
    \end{itemize}
    \item Solving polynomial equations.
    \begin{itemize}
     \item $solve([x^{2} - 4 x + 2 == -1], x)$ = $\sage{solve([x^2 - 4*x + 2 == -1], x)}$
     \item Solutions to $x^{2}+ 3 x y + y^{2} = 0$ and $x - y = 4$ = $\sage{map(lambda z: [z[0].rhs().n(30), z[1].rhs().n(30)], solve([x^2+3*x*y+y^2 == 0, x - y == 4], x, y))}$
    \end{itemize}
    \item Use find\_root where solve does not work. Also useful to find solutions in a particular interval.
    \begin{itemize}
     \item $solve(cos(t) == sin(t), t)$ = $\sage{solve(cos(t) == sin(t), t)}$
     \item $find \textunderscore root(cos(t) == sin(t), 0, pi)$ = $\sage{find_root(cos(t) == sin(t), 0, pi)}$
    \end{itemize}
  \end{itemize}
\end{frame}

\begin{frame}{Number Theory}
 \begin{itemize}
  \item Modulus operations
  \begin{itemize}
   \item $ring = IntegerModRing(12); ring(27)$ gives $\sage{ring(27)}$
   \item $mod(27, 12)$ = $\sage{mod(27, 12)}$ and $power \textunderscore mod(27, 2, 12)$ = $\sage{power_mod(27, 2, 12)}$
  \end{itemize}
  \item Primality test: $is \textunderscore prime(13)$ = $\sage{is_prime(13)}$, $is \textunderscore prime(15)$ = $\sage{is_prime(15)}$
  \item $prime \textunderscore range(1, 35)$ = $\sage{prime_range(1, 35)}$.
  \begin{itemize}
    \item Generator version: $primes(1, 35)$
  \end{itemize}
  \item Count of primes: $prime \textunderscore pi(35)$ = $\sage{prime_pi(35)}$
  \item $primes \textunderscore first \textunderscore n(11)$ = $\sage{primes_first_n(11)}$
  \item $next \textunderscore prime(29)$ = \sage{next_prime(29)} and $previous \textunderscore prime$ = \sage{previous_prime(29)}
  \item $factorial(20)$ = $\sage{factorial(20)}$, $factor(20)$ = $\sage{factor(20)}$, $divisors(20)$  = $\sage{divisors(20)}$
 \end{itemize}
\end{frame}

\begin{frame}{Graph Plotting}
  \begin{center}
    Circle of radius 4 centered at (0, 0): $c = circle((0, 0), 4)$
    \sageplot{plot(c, figsize=4)}
  \end{center}
\end{frame}

\begin{frame}{Graph Plotting(cont.)}
  \begin{center}
    Multiple functions in same plot:$plot(sin(x), -20, 20, rgbcolor = (0, 0, 1)) +$ \\ $plot(cos(x), -20, 20, rgbcolor = (1, 0, 0))$
    \sageplot{plot(sin(x), -20, 20, rgbcolor = (0, 0, 1)) + plot(cos(x), -20, 20, rgbcolor = (1, 0, 0), figsize=3.5)}
  \end{center}
\end{frame}

\begin{frame}{Graph Plotting(cont.)}
  \begin{center}
    $f = \frac{sin(y*y+x*x)}{\sqrt{(x*x+y*y+.0001)}}$: $plot3d(f, (-3, 3), (-3, 3))$
    \sageplot{plot3d(f,(-3,3),(-3,3), adaptive=True, color=rainbow(60, 'rgbtuple'), max_bend=.1, max_depth=15, figsize=2)}
  \end{center}
\end{frame}

\end{document}