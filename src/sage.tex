\documentclass{beamer}

\mode<presentation>
{
  \usetheme{Hannover}
  \usecolortheme{whale}
  \setbeamercovered{transparent}
}

\usepackage{amsmath}
\usepackage[english]{babel}
\usepackage{hyperref}
\usepackage[latin1]{inputenc}
\usepackage{moresize}
\usepackage{times}
\usepackage[T1]{fontenc}
\usepackage{sagetex}
\usepackage{upgreek}

\hypersetup{colorlinks=true, urlcolor=blue}

\title{An Introduction to Sage}

\author[Arvind]
{Arvind S Raj}

\institute[Amrita]
{
  Department of Cybersecurity Systems and Networks\\
  Amrita University, India
}

\date[FOSDEM 2014]
{1 February 2014 / FOSDEM}

\AtBeginSubsection[]
{
  \begin{frame}<beamer>{Outline}
    \tableofcontents[currentsection, sectionstyle=show/show, subsectionstyle=show/shaded/hide]
  \end{frame}
}

\AtBeginSection[]
{
  \begin{frame}<beamer>{Outline}
    \tableofcontents[currentsection, sectionstyle=show/show, subsectionstyle=show/shaded/hide]
  \end{frame}
}

\begin{document}

\begin{frame}
  \titlepage
\end{frame}

\begin{frame}{Outline}
  \tableofcontents[hideallsubsections]
\end{frame}

\begin{frame}{About me}
  \begin{itemize}
    \item Graduate CS student at Amrita University, India.
    \item Passionate about computer security and Python.
    \item Use Sage in Cryptography labs, Mathematics courses and CTF contests.
  \end{itemize}
\end{frame}

\section{Overview and Installation of Sage}
\begin{frame}{Getting Sage}
  \begin{itemize}
   \item Installation
    \begin{itemize}
     \item Pre-built binaries for most OS.
     \item PPA for Ubuntu.
     \item Packaging efforts underway for Debian and Fedora.
    \end{itemize}
   \item GPL licensed mathematics software.
   \item Unified interface to about 90 popular Python libraries.
   \item Two modes: command(like Python shell) and notebook(web interface).
   \item Power of IPython shell and Python programming language.
   \item ``sagerc`` file: \$HOME/.sage/init.sage or \$SAGE\_STARTUP\_FILE.
  \end{itemize}
\end{frame}

\section{Basic usage}
\subsection{Interactive shell and scripting}
\begin{frame}{Interactive shell and scripting}
  \begin{itemize}
   \item \textbf{Sage interpreter}: IPython shell.
   \item \textbf{Sage scripts}
   \begin{itemize}
    \item Similar to Python scripts; .sage extension.
    \item import names from sage.all
    \item Run as sage <filename> <arguments> like Python.
    \item Other possibilities: profiling, compiling sage files(Cython), access C functions directly.
   \end{itemize}
  \end{itemize}
\end{frame}

\subsection{Arithmetic and built-in functions}
\begin{frame}[fragile]{Arithmetic and built-in functions}
  \begin{itemize}
   \item General arithmetic supported by an (I)Python shell.
   \begin{itemize}
    \item \textasciicircum \hspace{0pt} is exponent and \textasciicircum\textasciicircum \hspace{0pt} is XOR.
    \item For integers, / reduces to lowest fraction and // performs integer division.
   \end{itemize}
   \item Support mathematical functions and constants with arbitrary precision.
   \begin{itemize}
    \item \verb+pi.n(digits=20)+ = \sage{pi.n(digits=20)}
    \item \verb+e.n(digits=25)+ = \sage{e.n(digits=25)}
    \item \verb+golden_ratio.n(prec=60)+ = \sage{golden_ratio.n(prec=60)}
    \item \verb+n(sin(pi/3), prec=60)+ = \sage{n(sin(pi/3), prec=60)}
    \item \verb+sqrt(263).n(digits=20)+ = \sage{sqrt(263).n(digits=20)}
    \item \verb+n(cos(5*pi/4), prec=60)+ = \sage{n(cos(5*pi/4), prec=60)}
   \end{itemize}
  \end{itemize}
\end{frame}

\begin{sagesilent}
 var('x y t')
 c = circle((0, 0), 4)
 ring = IntegerModRing(12)
 def f(x,y):
  return math.sin(y*y+x*x)/math.sqrt(x*x+y*y+.0001)
 P = Matrix([[1, 2], [3, 4]])
 Q = Matrix([[7, 8], [5, 6]])
 
 def escape(x):
  return str(x)

\end{sagesilent}

\section{Applications in various domains}
\subsection{Algebra}
\begin{frame}{Algebra}
  \begin{itemize}
    \item Factorizing polynomials.
    \begin{itemize}
      \item $factor(x^{4} - 15 x^{3} + 84 x^{2} - 208 x + 192)$ = $\sage{factor(x^4 - 15*x^3 + 84*x^2 - 208*x + 192)}$
      \item $factor(x^{3} - 6 x^{2} +11 x -6)$ = $\sage{factor(x^3 - 6*x^2 +11*x -6)}$
    \end{itemize}
    \item Solving polynomial equations.
    \begin{itemize}
     \item $solve([x^{2} - 4 x + 2 == -1], x)$ = $\sage{solve([x^2 - 4*x + 2 == -1], x)}$
     \item Solutions to $x^{2}+ 3 x y + y^{2} = 0$ and $x - y = 4$ = $\sage{map(lambda z: [z[0].rhs().n(30), z[1].rhs().n(30)], solve([x^2+3*x*y+y^2 == 0, x - y == 4], x, y))}$
    \end{itemize}
    \item Use find\_root where solve does not work. Also useful to find solutions in a particular interval.
    \begin{itemize}
     \item $solve(cos(t) == sin(t), t)$ = $\sage{solve(cos(t) == sin(t), t)}$
     \item $find \_ root(cos(t) == sin(t), 0, pi)$ = $\sage{find_root(cos(t) == sin(t), 0, pi)}$
    \end{itemize}
  \end{itemize}
\end{frame}

\subsection{Number Theory}
\begin{frame}{Number Theory}
 \begin{itemize}
  \item Modulus: $mod(27, 12)$ = $\sage{mod(27, 12)}$ and $power \_ mod(27, 2, 12)$ = $\sage{power_mod(27, 2, 12)}$
  \item Primality test: $is \_ prime(13)$ = $\sage{is_prime(13)}$, $is \_ prime(15)$ = $\sage{is_prime(15)}$
  \item $prime \_ range(1, 35)$ = $\sage{prime_range(1, 35)}$.
  \begin{itemize}
    \item Generator version: $primes(1, 35)$
  \end{itemize}
  \item $primes \_ first \_ n(11)$ = $\sage{primes_first_n(11)}$
  \item $next \_ prime(29)$ = $\sage{next_prime(29)}$ and $previous \_ prime$ = $\sage{previous_prime(29)}$
  \item $factorial(20)$ = $\sage{factorial(20)}$, $factor(20)$ = $\sage{factor(20)}$, $divisors(20)$  = $\sage{divisors(20)}$
  \item $gcd(10, 15)$ = $\sage{gcd(10, 15)}$, $lcm(10, 15)$ = $\sage{lcm(10, 15)}$
 \end{itemize}
\end{frame}

\subsection{Calculus}
\begin{frame}{Calculus}
  \begin{itemize}
    \item Differentiation
    \begin{itemize}
      \item $diff(sin(x) + cos(x)$ = $\sage{diff(sin(x) + cos(x))}$
      \item $diff((sin(x \textasciicircum 2) \textasciicircum 3))$ = $\sage{diff((sin(x^2)^3))}$
    \end{itemize}
    \item Integration
    \begin{itemize}
      \item $integral(cos(x) - sin(x))$ = $\sage{integral(cos(x) - sin(x))}$
      \item $integral(6*x*cos(x \textasciicircum 2)*sin(x \textasciicircum 2) \textasciicircum 2, x)$ = $\sage{integral(6*x*cos(x^2)*sin(x^2)^2)}$
    \end{itemize}
    \item Partial differential and solving differential equations also possible!
  \end{itemize}
\end{frame}

\subsection{Graph plotting}
\begin{frame}{Graph Plotting}
  \begin{center}
    Circle of radius 4 centered at (0, 0): $c = circle((0, 0), 4)$
    \sageplot{plot(c, figsize=4)}
  \end{center}
\end{frame}

\begin{frame}{Graph Plotting(cont.)}
  \centering Multiple functions in same plot.
  \begin{center}
    $plot(sin(x), -20, 20, rgbcolor = (0, 0, 1)) +$ \\ $plot(cos(x), -20, 20, rgbcolor = (1, 0, 0))$
    \sageplot{plot(sin(x), -20, 20, rgbcolor = (0, 0, 1)) + plot(cos(x), -20, 20, rgbcolor = (1, 0, 0), figsize=3.5)}
  \end{center}
\end{frame}

\begin{frame}{Graph Plotting(cont.)}
  \begin{center}
    $f = \frac{sin(y*y+x*x)}{\sqrt{(x*x+y*y+.0001)}}$: $plot3d(f, (-3, 3), (-3, 3))$
    \sageplot{plot3d(f,(-3,3),(-3,3), adaptive=True, color=rainbow(60, 'rgbtuple'), max_bend=.1, max_depth=15, figsize=2)}
  \end{center}
\end{frame}

\subsection{Matrix algebra}
\begin{frame}{Matrix algebra}
  \begin{itemize}
   \item Creating matrices: $m = Matrix([[1,2],[3,4],[5,6]])$
   \item Arithmetic operations
   \begin{itemize}
    \item $P = Matrix([[1, 2], [3, 4]])$, $Q = Matrix([[7, 8], [5, 6]])$
    \item P + Q = $\sage{P + Q}$, P - Q = $\sage{P - Q}$
    \item P * Q = $\sage{P * Q}$, 4 * P = $\sage{4 * P}$
   \end{itemize}
   \item $P^{3}$ = $\sage{P^3}$, $P^{-1}$ = $\sage{P^-1}$, |P| = $\sage{P.det()}$
   \item More functions: is\_singular, is\_symmetric, is\_skew\_symmetric, is\_invertible, is\_square
  \end{itemize}
\end{frame}

\subsection{Sage and \LaTeX}
\begin{frame}{\LaTeX and Sage\TeX}
 \begin{itemize}
   \item \LaTeX representation: latex(P) $\sage{escape(latex(P))}$
   \item view(P): Display PDF(pdflatex)/HTML(MathJAX) depending on mode.
   \item Sage\TeX: Call Sage commands from \LaTeX.
   \begin{itemize}
    \item Regular statement: \textbackslash sage\{pow\_mod(27, 2, 12)\}
    \item Plots: \textbackslash sageplot\{plot(sin(x) + cos(x), -20, 20)\}
    \item \textbackslash sageblock and \textbackslash sagesilent: Embedding Sage code
   \end{itemize}
 \end{itemize}
\end{frame}

\section{More applications and further reading}
\begin{frame}{Other applications}
 \begin{itemize}
  \item Interfacing with other algebra systems(GP/PARI, Singular, Maxima)
  \item Polynomials
  \item Combinatorics
  \item Graph and group theory
  \item Linear algebra
  \item Elliptic curves
  \item Advanced portions of everything discussed
 \end{itemize}
\end{frame}

\begin{frame}{References and further reading}
 \begin{itemize}
  \item Sage tutorial: \href{http://www.sagemath.org/doc/tutorial/index.html}{http://www.sagemath.org/doc/tutorial/index.html}
  \item Thematic tutorials: \href{http://www.sagemath.org/doc/thematic\_tutorials/index.html}{http://www.sagemath.org/doc/thematic\_tutorials/index.html}
  \item Tutorials for those with some mathematics background: \href{http://www.sagemath.org/doc/prep/index.html}{http://www.sagemath.org/doc/prep/index.html}
 \end{itemize}
\end{frame}

\section{Contributing to Sagemath}
\begin{frame}{Contributing to Sagemath}
 \begin{itemize}
  \item Packaging for Linux distros.
  \item Improve startup time.
  \item UI enhancements: Notebook and 2D plots.
  \item Mobile applications: Android, iOS.
  \item Mathematicians help with specific libraries.
  \item Visit \href{http://www.sagemath.org/development.html}{http://www.sagemath.org/development.html} for more information on getting involved.
 \end{itemize}
\end{frame}

\section{Questions?}
\begin{frame}
  \begin{center}
    \HUGE Questions?
  \end{center}
\end{frame}

\begin{frame}
  \begin{center}
    \HUGE Thank you!
  \end{center}
\end{frame}

\end{document}
