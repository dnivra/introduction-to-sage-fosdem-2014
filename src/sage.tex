\documentclass{beamer}

\mode<presentation>
{
  \usetheme{Warsaw}
  \setbeamercovered{transparent}
}

\usepackage{amsmath}
\usepackage[english]{babel}
\usepackage[latin1]{inputenc}
\usepackage{times}
\usepackage[T1]{fontenc}
\usepackage{sagetex}
\usepackage{upgreek}

\title{An Introduction to Sage}

\author[Arvind]
{Arvind S Raj}

\institute[Amrita]
{
  Department of Cybersecurity Systems and Networks\\
  Amrita University, India
}

\date[FOSDEM 2014]
{1 February 2014 / FOSDEM}

\AtBeginSubsection[]
{
  \begin{frame}<beamer>{Outline}
    \tableofcontents[currentsection,currentsubsection]
  \end{frame}
}

\begin{document}

\begin{frame}
  \titlepage
\end{frame}

\begin{frame}{Outline}
  \tableofcontents
\end{frame}

\begin{frame}{Bio}
  \begin{itemize}
    \item Master's student at Amrita University, India.
    \item Interested in computer security and Python.
    \item Used Sage to graduate labs in Cryptography and in CTF contests.
  \end{itemize}
\end{frame}

\begin{frame}{Introduction to Sage}
  \begin{itemize}
   \item GPL licensed mathematics software.
   \item Unified interface to about 90 popular Python libraries.
   \item Two modes: command(like Python shell) and notebook(web interface).
   \item IPython shell and Python programming language.
   \item ``sagerc`` file: \$HOME/.sage/init.sage or \$SAGE\_STARTUP\_FILE.
  \end{itemize}
\end{frame}

\begin{frame}[fragile]{Arithmetic and built-in functions}
  \begin{itemize}
   \item General arithmetic supported by an (I)Python shell.
   \begin{itemize}
    \item \textasciicircum \hspace{0pt} is exponent and \textasciicircum\textasciicircum \hspace{0pt} is XOR.
    \item For integers, / reduces to lowest fraction and // performs integer division.
   \end{itemize}
   \item Support mathematical functions and constants with arbitrary precision.
   \begin{itemize}
    \item \verb+pi.n(digits=20)+ = \sage{pi.n(digits=20)}
    \item \verb+e.n(digits=25)+ = \sage{e.n(digits=25)}
    \item \verb+golden_ratio.n(prec=60)+ = \sage{golden_ratio.n(prec=60)}
    \item \verb+n(sin(pi/3), prec=60)+ = \sage{n(sin(pi/3), prec=60)}
    \item \verb+sqrt(263).n(digits=20)+ = \sage{sqrt(263).n(digits=20)}
    \item \verb+n(cos(5*pi/4), prec=60)+ = \sage{n(cos(5*pi/4), prec=60)}
   \end{itemize}
  \end{itemize}
\end{frame}

\end{document}